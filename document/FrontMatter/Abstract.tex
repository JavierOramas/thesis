\begin{resumen}
	En la actualidad, la gestión eficiente de la vasta cantidad de documentos científicos vinculados a un tema específico se ha convertido en un desafío creciente. Este estudio propone un \emph{pipeline} diseñado para optimizar dicho proceso, abordándolo en tres etapas esenciales. Primero, se enfoca en la detección de los aspectos más relevantes dentro de la bibliografía proporcionada. Luego, se procede a la generación de un resumen abstractivo que condensa la información asociada con cada aspecto identificado, integrando referencias a los documentos pertinentes para respaldar la información presentada. Finalmente, se realiza una abstracción general del texto generado, proporcionando una visión consolidada de la información extraída.

	La evaluación de este enfoque se lleva a cabo mediante el análisis de resultados utilizando conjuntos de datos específicamente diseñados para el procesamiento de múltiples documentos. Este proceso de evaluación busca validar la efectividad y la precisión del \emph{pipeline} propuesto en la identificación y síntesis de información relevante a partir de documentos científicos extensos. Con esta investigación, se busca no solo mejorar la eficiencia en la revisión de la literatura científica, sino también proporcionar una herramienta valiosa para investigadores y profesionales que enfrentan el desafío de abordar grandes volúmenes de información en sus respectivos campos.
\end{resumen}

\begin{abstract}
	Currently, the efficient management of the vast amount of scientific documents associated with a specific topic has become an increasingly challenging task. This study proposes a \emph{pipeline} designed to streamline this process, addressing it in three essential stages. First, it focuses on detecting the most relevant aspects within the existing literature. Next, it proceeds to generate an abstractive summary that condenses information associated with each identified aspect, integrating references to pertinent documents to support the presented information. Finally, a general abstraction of the generated text is performed, providing a consolidated view of the extracted information.

	The evaluation of this approach is carried out by analyzing results using datasets specifically designed for processing multiple documents. This evaluation process aims to validate the effectiveness and accuracy of the proposed \emph{pipeline} in identifying and synthesizing relevant information from extensive scientific documents. With this research, the goal is not only to improve efficiency in reviewing scientific literature but also to provide a valuable tool for researchers and professionals facing the challenge of addressing large volumes of information in their respective fields.
\end{abstract}