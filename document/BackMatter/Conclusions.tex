\begin{conclusions}
    El presente trabajo de grado ha introducido un modelo capaz de generar borradores de estados del arte utilizando un conjunto preseleccionado de documentos. Este modelo demuestra eficacia al producir un sistema de citaciones coherente con la información contenida en los documentos de entrada. Los resultados experimentales indican que es posible obtener un estado del arte sobre un tema específico en menos de 24 horas, lo cual aborda de manera efectiva la problemática asociada al tiempo que requiere realizar una investigación en la actualidad.

    Los experimentos realizados para validar el modelo con conjuntos de datos reales han arrojado resultados prometedores, respaldando la viabilidad y eficacia de la propuesta. No obstante, es imperativo reconocer las consideraciones especiales y limitaciones inherentes a esta investigación.
    
    Una limitación identificada radica en que, al pasar el sistema de citación por el \emph{LLM}, la consistencia se ve afectada, es por ello que se decide conservar la estructura generada por el resumen extractivo, provocando citas más extensas pero también mas precisas.
    
    Adicionalmente, se destaca la posible limitación del hardware para la implementación óptima del modelo. La exigencia de hardware especializado puede constituir un obstáculo, especialmente en relación con el factor tiempo. Esto puede llegar a implicar la imposibilidad de utilizar el modelo propuesto si no se cuenta con \emph{hardware} especializado.
    
    Durante la fase experimental, se observó que la cantidad y diversidad de los documentos influyen en las métricas del modelo. La significación de esta variabilidad en los resultados requiere una evaluación más detallada.
    
    En conclusión, aunque este trabajo representa un avance sustancial en la automatización de la generación de estados del arte, se reconoce la necesidad de abordar las limitaciones mencionadas. Estas consideraciones ofrecen oportunidades para futuras investigaciones y mejoras.
\end{conclusions}
