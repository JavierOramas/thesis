\begin{recomendations}
    Se plantea como sugerencias fundamentales para futuras investigaciones la creación de conjuntos de datos específicos que superen las limitaciones actuales de los datos públicos, centrándose especialmente en la expansión del texto y la diversificación de las fuentes. Se aboga por la inclusión de conocimientos no científicos e información multilingüe con el objetivo de enriquecer la calidad de los conjuntos de datos. También se propone realizar un estudio cuantitativo para evaluar cómo la diversidad de contenido y el número de documentos impactan la calidad de los resúmenes, investigando posibles desviaciones en las métricas de evaluación.

    Además, se recomienda encarecidamente el desarrollo de una implementación que fomente una mayor interacción del usuario, concediéndole la capacidad de solicitar modificaciones en el documento ya sea total o parcialmente según su criterio. Esta recomendación incluye la implementación de una interfaz gráfica intuitiva y la integración de funcionalidades de preguntas y respuestas referentes a los documentos, con el propósito de mejorar la experiencia y brindar mayor control sobre el contenido generado.
    
    Asimismo, se destaca la importancia de asegurar el soporte a varios formatos de salida establecidos, como Latex, JSON, entre otros. Esta medida busca adaptarse a las diversas necesidades de los usuarios y facilitar la interoperabilidad con otras herramientas y plataformas. Estas recomendaciones se presentan como contribuciones esenciales para avanzar en el desarrollo de herramientas de resumen de texto más efectivas y versátiles.

    Con el fin de mejorar los resultados del modelo se sugiere añadir un sistema de efvaluación de relevancia de un documento que tenga en cuenta los metadatos del documento, de modo que, se pueda otorgar más relevancia a documentos publicados por autoridades de mayor reputación u otros criterios de segregación.
    
\end{recomendations}
