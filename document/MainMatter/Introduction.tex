\chapter*{Introducción}\label{chapter:introduction}
\addcontentsline{toc}{chapter}{Introducción}

    El vertiginoso flujo de producción de artículos científicos en la actualidad plantea un desafío sustancial: la imposibilidad práctica de revisar exhaustivamente todos los documentos relacionados con una investigación. Solo en \emph{ArXiv}\footnote{ArXiv es un repositorio en línea donde los investigadores comparten artículos académicos antes de la revisión por pares y la publicación formal.}, se registran alrededor de 180,000 nuevas (pre) publicaciones por año [\cite{arxivstats}]. Esta abrumadora cantidad de información subraya la necesidad imperante de encontrar enfoques eficientes para obtener resúmenes de documentos científicos, una problemática que ha sido abordada a lo largo de los años mediante diversos intentos de automatización.

    La historia de los resúmenes automáticos se remonta a las décadas de 1950 y 1960, cuando la inteligencia artificial (\emph{IA}) comenzó a surgir como un campo de investigación. Desde este período, los científicos se centraron en el procesamiento del lenguaje natural (\emph{NLP} por sus siglas en inglés) como un componente clave para desarrollar sistemas de resumen. Uno de los hitos tempranos fue el trabajo de Hans Peter Luhn en 1958 [\cite{luhun1958}], quien propuso un método para extraer automáticamente oraciones clave de un documento basado en la frecuencia de las palabras clave. Este enfoque inicial estableció las bases para futuras investigaciones en resúmenes automáticos.

    Con el avance de las investigaciones surgieron enfoques más sofisticados: algoritmos basados en grafos, como \emph{TextRank} [\cite{mihalcea2004textrank}], los cuales se convirtieron en una herramienta valiosa para la extracción de información. Las investigaciones también han explorado enfoques supervisados [\cite{collins-etal-2017-supervised}], donde se utilizan técnicas de clasificación de oraciones, y se han abordado desafíos tanto en su comprensión como en la generación de lenguaje. [\cite{knight2000statistics}]

    Se ha experimentado un significativo avance con el desarrollo de los algoritmos de extracción y abstracción de información. Estos enfoques fundamentales definen cómo los sistemas automáticos condensan la información de un texto, ya sea seleccionando oraciones del texto original o generando un texto totalmente nuevo que contenga la información del original de manera más concisa.

    Los algoritmos de extracción, al seleccionar oraciones de la fuente de información, presentan ventajas notables. En principio, conservan la información original, y son especialmente útiles en contextos donde la fidelidad a los hechos es esencial, como en informes científicos o noticias. Además, al evitar la generación de lenguaje nuevo, estos enfoques minimizan el riesgo de errores gramaticales y son computacionalmente eficientes. Sin embargo, su limitación radica en la falta de creatividad y generalización, dependiendo exclusivamente de la información presente en el texto original, al tiempo que puede conducir a resúmenes redundantes [\cite{LexRank}]

    En oposición a la extracción, los algoritmos de abstracción buscan generar una redacción nueva, ofreciendo creatividad y capacidad de generalización. Esta capacidad de expresar la esencia del contenido de manera única es una ventaja clave [\cite{rush2015neural}]. Además, la abstracción puede manejar información no presente de manera literal en el texto original, permitiendo una expresión más amplia de ideas [\cite{PaulusXS17}]. Sin embargo, los modelos con este enfoque enfrentan desafíos significativos tanto en la generación de lenguaje como en la posibilidad de incurrir en errores gramaticales o falta de coherencia. También existe el riesgo de introducir sesgos, dependiendo de la calidad del modelo y de los datos de entrenamiento. La abstracción suele ser más compleja computacionalmente en comparación con la extracción.

    No obstante, existen también enfoques híbridos [\cite{SeeLM17}], ya sea integrando técnicas de extracción y abstracción, o  combinaciones de diferentes enfoques dentro de un marco unificado. Estos enfoques buscan aprovechar las fortalezas de cada método para mejorar la calidad y la diversidad de los resúmenes generados.

    Este enfoque más amplio no solo aborda la creciente complejidad asociada con la información dispersa, sino que también presenta potencialidades para mejorar la coherencia temática y la comprensión integral de un tema al considerar múltiples perspectivas y contextos.

    Históricamente, la mayoría de los enfoques en el campo del resumen automático se han orientado hacia la generación de resúmenes para un solo documento, buscando condensar la información esencial de manera concisa. Sin embargo, a medida que la era digital ha transformado la forma en que accedemos a la información, la investigación más reciente ha dirigido su atención hacia la generación de resúmenes de múltiples documentos. Este cambio estratégico se ha vuelto esencial en entornos donde la información relevante se distribuye en diversas fuentes, como es el caso de la generación de resúmenes de noticias [\cite{fabbri2019multi-news}], la síntesis de artículos de Wikipedia [\cite{liu2018}], y la producción de resúmenes tipo \emph{Estado del arte}\footnote{Un estado del arte es una revisión exhaustiva y actualizada de los avances, investigaciones y desarrollos existentes en un campo específico de estudio o disciplina, que proporciona una visión panorámica de la situación actual del conocimiento en ese ámbito. } para documentos científicos [\cite{lu2020multixscience}].

    La generación de resúmenes de múltiples documentos no se limita simplemente a la agregación de información; más bien, implica la síntesis de conocimientos interconectados para ofrecer una visión holística. 
    Estos avances reflejan la adaptación dinámica de los métodos de resumen automático para satisfacer las demandas cambiantes de la era digital y para abordar la abundancia de información interrelacionada que caracteriza el entorno actual de la investigación y la comunicación científica.

    Los Modelos de Lenguaje con Aprendizaje Profundo (LLM, por sus siglas en inglés) han emergido como elementos cruciales para la generación de resúmenes automáticos. Los LLM, han demostrado una capacidad excepcional para entender la semántica y la estructura del lenguaje [\cite{fewshot}]. Su entrenamiento masivo en grandes corpora\footnote{Conjunto de datos, textos u otros materiales sobre determinada materia que pueden servir de base para una investigación o trabajo.} de texto les permite capturar patrones complejos y generar resúmenes coherentes y contextualmente relevantes.

    Las ventajas de los LLM en este contexto son notables. En primer lugar, su capacidad para comprender el significado contextual permite generar resúmenes que no solo seleccionan información clave, sino que también la expresan de manera más fluida y comprensible [\cite{Radford2018ImprovingLU}]. Además, los LLM pueden abordar la generación de lenguaje nuevo de manera más efectiva, superando algunos desafíos asociados con la abstracción, como errores gramaticales y falta de coherencia [\cite{RoBERTa}].

\section{Motivación}
    Es evidente la necesidad de crear una herramienta que permita a los investigadores procesar grandes cantidades de texto de manera eficiente y rebasar las limitaciones que actualmente suponen tanto el tiempo de lectura de artículos científicos como el volumen de información sobre cada tema para la creación de estados del arte en investigaciones científicas

    Si bien existen algunos sistemas  [\cite{elicit},\cite{scite}], que permiten la extracción de información de múltiples documentos, por lo general requieren de un conocimiento previo de los temas que tratan. Además, la interacción en escencia se presenta en formato de preguntas-respuestas que, si bien puede ser deseable en algunos casos, puede hacer más tedioso el proceso de obtener de manera clara la información que se desea. Además, estos sistemas dependen de la correcta comprensión por parte del modelo de las preguntas que se le realizan.\par

\section{Problemática}
    Al hacer un estudio del estado del arte en torno a los sistemas de generación de resúmenes enfocados a la construcción de los propios estados del arte, no se encontraron modelos o metodologías públicas que sean capaces de generar textos semántica y sintácticamente correctos, y que a su vez cuenten con un sistema de citación robusto que cumpla con los requisitos de un estado del arte. Todos los candidatos encontrados resultaron deficientes en, al menos, una categoría.

\section{Hipótesis}
    Es posible desarrollar un sistema (herramienta) de generación de un borrador de estado del arte que tenga como base de conocimientos un grupo de documentos previamente seleccionados y que, a su vez, genere un texto resumen que abarque un grupo de los temas (autodetectados) más relevantes. Además, el sistema deberá ser capaz de generar citas bibliográficas de manera automática, correcta y en el formato adecuado, cumpliendo con una estructura determinada.

\section{Objetivos}
    \subsection{Objetivo General}

        Proponer un modelo que permita generar un texto que contenga una enumeración de los principales temas encontrados en el corpus de documentos; seguidos cada uno por un resumen de los contenidos `relacionados con el tema' presentes en el \emph{dataset} y con sus respectivas citaciones.

    \subsection{Objetivos Específicos}

        \begin{itemize}
            \item Definir la estructura que deberán cumplir todos los documentos generados.
            \item Proponer una metodología para la extracción de los temas principales de los documentos.
            \item Identificar o entrenar un modelo que se desempeñe satisfactoriamente en la generación de un texto que cumpla con la estructura predefinida.
            \item Realizar una validación de los resultados que permita verificar la viabilidad de el método sugerido para la generación de un texto que cumpla con los requisitos definidos.
        \end{itemize}
\section{Esbozo de solución}
    En este documento se hace una propuesta de modelo híbrido, que utiliza la representación mediante \emph{embeddings}\footnote{Un \emph{embedding} de texto es una técnica que representa palabras como vectores de números, permitiendo que palabras con significados similares tengan representaciones similares}
    de los documentos para obtener una representación vectorial del corpus, sobre la cual se trabajará para agrupar los documentos por temas. Una vez hecho esto, se obtendrán las ideas más relevantes y, mediante la utilización de un modelo de lenguaje, se modelará la información extraída para cumplir con el formato especificado.

\section{Estructura de la tesis}
        El resto del documento se encuentra organizado de la siguiente manera:

        \begin{enumerate}
            \item \emph{Capítulo 1} Se hace una revisión del estado del arte en el ámbito de la generación de resúmenes. Principalmente buscando enfoques híbridos, y modelos (o metodologías) que reduzcan los inconvenientes de los modelos clásicos.
            \item \emph{Capítulo 2} Se describe la propuesta de solución, en la que se detalla la metodología utilizada en la generación de los resúmenes.
            \item \emph{Capítulo 3} Se presentan los detalles de implementación y retos computacionales afrontados, así como la discusión y comparación de los resultados obtenidos.
            \item \emph{Capítulo 4} Se presentan conclusiones y recomendaciones para futuros trabajos.
            \item \emph{Capítulo 5} Se presentan bibliografías y anexos relacionados con la temática abordada.
        \end{enumerate}


